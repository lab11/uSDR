\chapter{Demodulation of 802.15.4}
\section{Demodulation equation}
The receiver uses the following equation to demodulate the signal.
\begin{equation}
d(t) = R_Q(t)\times R_I(t-\tau) - R_Q(t-\tau)\times R_I(t)
\end{equation}

where the $R_I(t)/R_Q(t)$ are received In-phase/Quad-phase signals, $\tau$ is the sampling period. Bit decoding depends on
the sign of $d(t)$. If $d(t)>0$, the decoded bit equals 1 and vice versa.

\section{Decoding principle}
The 802.15.4 standard uses OQPSK modulation. The In-phase and Quad-phase are shifted by 1/2 symbol time. The symbol time 
is 1/2 period of sine wave. Namely, the In-phase and Quad-phase are shifted by 1/4 period of sine wave. Thus, the In-phase
and Quad-phase can be expressed as $\pm sin(t + \theta)$ and $\pm cos(t + \theta)$. Where $\theta \in \{0, \tfrac{\pi}{2}\}$.

For example, one combination of In-phase/Quad-phase is $cos(t)$ and $sin(t), 0<t<\tfrac{pi}{2}$. By applying the demodulation
equation,
\begin{align}
d(t)&= sin(t)\times cos(t-\tau) - sin(t-\tau)\times cos(t) \\
d(t)&= sin(t)\left[cos(t)cos(\tau) + sin(t)sin(\tau)\right] - cos(t)\left[sin(t)cos(\tau) - cos(t)sin(\tau)\right] \\
d(t)&= sin^2(t)sin(\tau) + cos^2(t)sin(\tau) = sin(\tau)
\end{align}
Therefore, $d(t)$ depends on the sampleing period only. In this example, $d(t)$ is greater than 0 and remains constant 
during the half-symbol period. The following table lists the total combination of I/Q signals.

\begin{table}[h!]
\centering
	\begin{tabular}{|c|c|c|}
		\hline
		In-phase & Quad-phase 	& decoded equation \\ \hline	
		$cos(t)$ & $sin(t)$		& $sin(\tau)$\\ \hline
		$cos(t)$ & $-sin(t)$	& $-sin(\tau)$\\ \hline
		$-cos(t)$ & $sin(t)$	& $-sin(\tau)$\\ \hline
		$-cos(t)$ & $-sin(t)$	& $sin(\tau)$\\ \hline
		$sin(t)$ & $cos(t)$		& $-sin(\tau)$\\ \hline
		$sin(t)$ & $-cos(t)$	& $sin(\tau)$\\ \hline
		$-sin(t)$ & $cos(t)$	& $sin(\tau)$\\ \hline
		$-sin(t)$ & $-cos(t)$	& $-sin(\tau)$\\ \hline
	\end{tabular}
	\label{tab:demux}
	\caption{}
\end{table}

Therefore, the receiver simply determines the incoming bit from the sign of $d(t)$. Note that the I/Q signals coming
at the symbol rate of 1$\mu$s, and the I/Q shift by 0.5$\mu$s. Every 0.5$\mu$s generates a bit. Bit rate is 2MHz and 
half-byte is mapped to 32 bits chip sequence, which is 16$\mu$s.
