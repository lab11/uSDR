\section{Conclusions}
\label{sec:conc}

Software-defined radios are reconfigurable communication systems that
transcend historical boundaries between hardware and software subsystems,
physical and logical layers, and analog and digital domains.  In so doing,
they enable radical new architectures, novel radio designs, and
high-performance protocols that are not easy to design, implement, or evaluate
using traditionally-layered approaches.  Although modern SDR platforms have
been used to explore many facets of the wireless design space, their current
architectures make it very difficult to explore the {\em low-power} design
space.  Their use of SRAM-based FPGAs result in high static and dynamic power
draws, their slow startup times are not amenable to rapid duty cycling, their
radio front-ends do not support power controls, and their processing
requirements place a heavy load on the system.  As a result, fertile
application areas like mobile phones and sensor networks that could benefit
from radical approaches, but which require low-power operation, remain
relatively unexplored.

We developed \sdr to address this inequity. \sdr is a small, low-cost, and
low-power software-defined radio platform that leverages emerging technology
like highly-integrated radio front-ends and mixed-signal FPGA processing
back-ends. This paper demonstrates that a software radio with a footprint of
below 100~cm$^2$ and costing less than \$150 is able to operate from a
set of `AA' batteries, and that we can expect a battery-life similar to
today's smart phones. This work enables new research areas that were completely
out-of-reach or existed only in severely limited forms in low-power nodes.
Hence, \sdr is an enabling technology for many high-impact, large scale
applications of low-power, ad-hoc wireless networking where high performance
and/or precise timing are required, including full-duplex wireless
communication, synchronous concurrent communication, high-frequency power
metering, infrastructure less audio/video streaming, and structural health
monitoring.
