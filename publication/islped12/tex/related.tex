\section{Related Work}
\label{sec:related}

\begin{comment}
\begin{table*}
\centering
\begin{threeparttable}
	\begin{tabular}{|l|c|c|c|c|c|c|c|} \hline
		\rowcolor[gray]{0}
		% header line
		\multicolumn{1}{|>{\columncolor[gray]{0}}c|}{\sc {\color{white} Platform}} &
		{\sc {\color{white} Sleep}} &
		{\sc {\color{white} Power}} &
		{\sc {\color{white} Size }}&
		{\sc {\color{white} Interconnect}}&
		{\sc {\color{white} Throughput}} &
		{\sc {\color{white} Price}} &
		% XXX: Usage -> adoption? Or some such metric is the idea behind this column
		{\sc {\color{white} Realization}}
		\\ \hline
		% SORA cost is totally estimated? No reference in the paper, but > PC cost..
		SORA {\small \cite{sora}}	 & -	& $>$100~W\tnote{c}	 & 36000~cm$^{3}$\tnote{c} & PCI-Express	& 16.7~Gb/s	& \$2000\tnote{c} & Research \\ \hline
		% KUAR power Pentium M min 5W, size COM Express 125 × 95 x (2 assumed) = 237.5, round 240, price: says "low cost" 3 times, but no price :(
		KUAR {\small \cite{kuar}}	 & -	& $>$5~W		 & 240~cm$^{3}$\tnote{f} &PCI-Express	& 2~Gb/s	& - & Research \\ \hline
		SODA {\small (180nm)~\cite{soda}} & -& \realtilde3~W & 26.6~mm$^{3}$\tnote{e}& DMA\tnote{a}	& 24~Mb/s\tnote{b} & -\tnote{d} & Simulated \\ \hline
		SODA {\small (90nm)~\cite{soda}} & -& \realtilde0.5~W & 6.7~mm$^{3}$\tnote{e}	& DMA\tnote{a}	& 24~Mb/s\tnote{b} & -\tnote{d} & Theoretical \\ \hline
		% WARP (numbers from energy wiki, where orig?), area 20 x 20 x >1, but we don't know it for sure so estimate conservative 2cm
		% Orig NRG num 10~15W, +30W / daughter card: 10~130W
		WARP~{\small \cite{warp-platform}} & -	& 10\realtilde130~W	 & 800~cm$^{3}$\tnote{f}	& Parallel MGTs\tnote{g}	& 24~Gb/s	& \$9750	 & Research \\ \hline
		%AirBlue Not really a platform
		% USRP 2, power 1.3-2.3A * 6V, area 22 x 16 x 5 cm
		% rate to/from host 50 MHz recv, 25 MHz xmit, 50/25 MSPS also
		% we'll go generous on the reported value
		USRP 2\tnote{c}~~{\small \cite{usrp:n200}}& -  & 7.9\realtilde13.8~W\tnote{c}	& 1760~cm$^{3}$\tnote{c} & Ethernet		& 1~Gb/s	& \$1700\tnote{c} & Commercial \\ \hline
		% E100, power 1.5-2.5A * 6V, area 22 x 16 x 5 cm
		% idle power is ~900mA * 6V = 5.5W
		% rate to/from host 4MHz, 4MSPS
% OMAP 3 GPMC Bus throughput: (we'll use best possible)
% http://e2e.ti.com/support/dsp/omap_applications_processors/f/42/t/35182.aspx#123019
		USRP E100~{\small \cite{usrp:e100}}& 5.5~W	& 9\realtilde15~W		 & 1760~cm$^{3}$		& OMAP 3 GPMC & 1.3~Gb/s	& \$1300	 & Commercial \\ \hline
		% uSDR, area 13 x 7 x 2
		\rowcolor[gray]{.9}
		\sdr	 & 0.32~W	& 1.4~W			 & 182~cm$^{3}$	& AMBA		& 1.4~Gb/s	& \$150\tnote{h}	 & Research \\ \hline
	\end{tabular}
	\begin{tablenotes}
		\small
		\item [a] Memory bus architecture not specified in~\cite{soda}.
		\item [b] Inferred minimum, may be faster.
		\item [c] Requires a companion PC. Not factored in power, portability, or cost for the USRP 1/2.
		\item [d] SODA is a custom chip that would likely have an extremely high die cost, but low per-unit cost.
		\item [e] Assumes $1mm$ thick.
		\item [f] Assumes $2cm$ thick.
		\item [g] ``Multi-Gigabit Transceiver'', an interconnect technology built into Xilinx FPGAs. Uses up to 8 parallel 3~Gb/s transceivers
		\item [h] Assumes 1,000 unit production run. See Table~\ref{tab:cost} for detailed breakdown.
	\end{tablenotes}
	\caption{A comparison of SDR platforms. The range in power comes from
boards whose power usage varies depending on the presence and type of daughter
card installed in the system. Where possible a measured idle / sleep power is
also shown.  For platforms that only list area we make reasonable assumptions
on height. \sdr is 10\% the cost of the next most expensive SDR platform, yet
provides parable speeds in the smallest non-IC package. It uses less power
than any realized hardware and nearly ties the previous best theoretical
hardware.}
	\label{tab:comparison}
\end{threeparttable}
\end{table*}
\end{comment}

While there are a diverse array of SDR platforms~(\cite{warp-platform,
  soda, kuar, sora}), none of the current platforms are suitable for low
power radio research and development. In particular, \sdr optimizes
for power, price, and portability without appreciably sacrificing
flexibility or usability. A comparison to previous SDR platforms is
shown in Table~\ref{tab:comparison}.

\subsection{Throughput and Latency}
\label{sec:related-throughput}
% PCI-Express up to 25W max draw
As Schmid identifies in~\cite{schmid-latency}, a strongly limiting
factor for any SDR system is the available throughput and latency of
the interconnect between the ``hard'' and ``soft'' portions of the SDR
platform. As this latency defines the critical path for the control
loop, previous work places significant focus on minimizing it.

In developing SORA, the PCI-Express bus was selected for its bounded
latency and high throughput, at the cost
of its high power requirements and restricting the platform to the PC
form factor. This design affords SORA the support of a complete PC
operating system for multi-tasking and control, at the cost of
run-time adaptivity.
%
Ettus's USRP platform also relies on the PC operating system for much
of its command and control functionality, however it eschews the fixed
form-factor of the PCI-E bus, instead relying on more conventional
peripheral buses: USB or Ethernet. While these
buses do provide high throughput, Nychis et al. find their
latency to be highly variable, as high as
9000$\mu$s~\cite{cmu-mac-sdr}. For the development of low-power
protocols, such high and variable latency is untenable.

\begin{comment}
Ettus's USRP platform also relies on a controlling PC, however it trades
throughput for flexibility. The USRP is a run-time peripheral, utilizing
either USB (480~Mbits/s) or Ethernet (1~Gb/s) as a interconnect. The effective
rate, however, is limited by hardware components to 50~MHz receiving and 25~MHz
transmitting).  Nychis further shows in~\cite{cmu-mac-sdr} that the
variable latency of the USB and Ethernet interfaces further reduce the
effective throughput when reliable latency is required - a key component in
the development of low-power protocols. Ettus's embedded E100 platform suffers
from the same limitations, running a full Linux instance on the embedded
Gumstix platform to interface with the software radio over the same
channels~\cite{ettus}. % XXX: Did we ever prove this?
\end{comment}
%
% Okay SODA, I quit, I do not know how to reasonably compare your throughput /
% latency to any other platform. Intra-core speed is comparable to the
% processing power of the FPGA (when all cores are combined), inter-core
% throughput is probably best, but they don't report this at all...
Instead then, \sdr follows in the footsteps of SODA, tightly coupling
the hardware compute engine with an ARM Cortex-M3 core~\cite{soda}. Both systems
allow for low latency (0.46~$\mu$s) and high throughput
(172~MB/s). 
%(though OS's for the M3 do exist~\cite{coocox,etc})
Neither SODA nor \sdr use an operating system in the traditional
sense, however as a custom micro architecture, SODA goes even
further. It defines a slightly customized VLIW+SIMD ISA and sacrifices
traditional memory consistency semantics for performance. Users of
SODA then must carefully hand-tune their programs not only to optimize
performance, but to even run correctly. SODA is not a full-featured
SDR platform, rather an advanced DSP architecture. While this
microarchitectural compromise allows for better performance per watt,
\sdr has been realized in existing commodity hardware (as opposed to
just simulation), with a measured full system power draw only slightly
higher than SODA's theoretical optimized form, as
Table~\ref{tab:comparison} shows.
% also would like to say something here about \sdr not yet being at all
% optimized for power (gating, etc)

\subsection{Power}
\label{sec:related-power}
We first divide existing
SDR platforms into two broad categories: those reliant on a PC or
PC-architecture and those that are stand alone. Power and portability
are not a design consideration for these PC-platforms (SORA and USRP
2), thus it is unsurprising that \sdr, along with the other embedded
platforms, are orders of magnitude better in these metrics.

Of the remaining platforms, \sdr, KUAR, WARP, USRP E100, and SODA,
\sdr excels in power draw and is competitive in size with all except
the custom chip SODA. Both KUAR and the USRP E100 can be considered
``near-PC'' platforms that embed low-power, PC-like components in a
compact form factor.  KUAR does not explicitly report power numbers,
but builds a custom board driven by a Pentium~M whose most efficient
model draws 5~W~\cite{pentium-m}, which we use as a generously
conservative approximation for its power draw.  The USRP E100
datasheet reports 15~W load with RF daughtercard installed. In our
power measurements, shown in more detail in
Section~\ref{sec:eval}, we find an idle power draw of
5.5~W. The WARP base system draws about 10~W, but is capable of
supplying up to 30~W to each of four daughter cards, allowing for a
peak power draw of 130~W.

All of these previous systems have one thing in common: Their power usage is
reported in {\em watts}.  With \sdr, we deliver the first realized {\em
sub-watt} SDR platform.  This is a significant milestone. \sdr is the first
SDR platform capable of running for a full day on a pack of AA
batteries.
\begin{comment}
~\footnote{Extrapolating from~\cite{mote-power}, we consider four AA
batteries with average of 2500~mAh and a fully charged nominal voltage of
1.6V~. The minimum supply voltage to \sdr is 4~V, yielding 9.75~Wh of
usable power. This is enough energy for \sdr to run 7.5 hours at full power
or 30.5 hours in maximum sleep.}
\end{comment}
% Assuming a roughly linear scaling as # of batteries increases
% 2500mAh * 4.8V = 12Wh = 43200Ws
% 6.4-5.6V -- 31.25% of 43200 -> 13500
% 5.6-4.8V -- 27.10% of 43200 -> 11707.2
% 4.8-4.0V -- 22.90% of 43200 ->  9892.8
% Don't care after 4V.  Total -> 35100Ws (9.75Wh)

\subsection{Portability}
\label{sec:related-portability}
As we enter the era of {\em sub-watt} SDR platforms, we find that the
size, and in turn portability, of the SDR platform becomes a critical
design consideration. The current \sdr design is approximately four
times the size of popular research motes such as the Mica, TelosB, or
Tmote Sky, shown in Figure~\ref{fig:usdr}. We note that with the
removal of the external memory controller, \sdr's extra interfaces
(such as Ethernet), and a slightly more compact layout, the \sdr
platform could easily halve its size.

\subsection{Price}
Perhaps the greatest advantage of the \sdr system compared to previous
work is its extremely low cost -- an order of magnitude less expensive
than other SDR platform. The first large cost reduction comes from
building a standalone platform, rather than relying on a costly
support PC (SORA, USRP2) or the slightly less costly embedded PC-like
environment (KUAR, USRP E100). The next biggest saving comes from
eschewing the more traditional daughtercard based SDR approach,
instead opting to design a dedicated 2.4~MHz RF frontend. We note that
while the \sdr system lacks physical daughtercards, it does not
completely sacrifice modularity. Separate RF frontends can be
``dropped in'' to the schematic and used interchangeably. As evidence,
collaborators at other institutions
% Can we say this?
are already designing a new board with a 5~GHz frontend. We also find
considerable cost-savings by using fewer and less-powerful FPGAs than
the WARP platform. Despite the significantly lower processing power of
the \sdr platform, it is sufficiently capable of supporting an
802.15.4 radio, and building the radio on a significantly smaller
budget.

\begin{comment}
\subsection{Deployability}
Ultimately, \sdr contributes what we argue to be the first truly
{\em deployable} SDR platform. Previous platforms were tied to PCs
(SORA, USRP 2), tied to large power supplies (KUAR, WARP, USRP E100),
or existed only as simulations (SODA). Even resolving these issues,
all prior platforms were prohibitively expensive for any large scale
deployment. We estimate the cost per node of a 1,000 node \sdr
deployment to be only $\$150$, a full order of magnitude better than
the previous state of the art.
\end{comment}


